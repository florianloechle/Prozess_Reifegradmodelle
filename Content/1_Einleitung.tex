\newpage
\fancyhead[L]{\textcolor{lightgrey}{\nouppercase{\rightmark}}}
\pagenumbering{arabic}

\section{Einleitung}

Beim Prozessmanagement geht es in erster Linie um die Planung, Steuerung und Überwa-chung von Unternehmensprozessen und zusätzlich auch um die
Aufträge auf strategischer, taktischer und operativer Ebene. Dadurch soll eine hohe Effizienz und Wirtschaftlichkeit in den Abläufen gewährleisten
werden (vgl. \cite[S.5]{Schmidt2012}). \par

Gleichzeitig ist ein Grundsatz des Prozessmanagements die kontinuierliche Verbesserung der vorhandenen Prozesse und die stetige Optimierung der Abläufe
innerhalb eines Unternehmens. Hier kommen zumehrt besondere Reifegradmodelle zum Einsatz, mit jenene Unternehmen die derzeitigen Istzustände erheben können. Darüber
unterstützen die Modelle bei der Aufdeckung und Identifikation von Verbesserungspotenzialen in der Prozess Struktur (vgl. \cite[S.1]{Kamprath2011}). \par

Reifegradmodelle, im englischen auch \glqq maturity model\grqq{} genannt, können Unternehmen bei der gezielten Verbesserung von Prozessen unterstützen. Nach Kamprath, sind
die Modelle ein hilfreiches Instrument im Prozessmanagement und helfen bei der Ermittlung des aktuellen Zustands im Unternehmen und bei der Ermittlung von Verbesserungspotenzialen (vgl. \cite[S.1]{Kamprath2011}). \par

Der Ursprung des Reifegradmodelle geht dabei zurück auf das Qualitätsmanagement. Dieses wird dazu verwendet Prozesse zu
bewerten (vgl. \cite[S.8]{Durr2008}).

\subsection{Aufgabe}
\subsection{Struktur der Arbeit}
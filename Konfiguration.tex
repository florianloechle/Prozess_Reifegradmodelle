%%%%%%%%%%%%%%%%%%%%%%%%%%%%%%%%%%%%%%%%%%%%%%%%%%%%%%%%%%%%%%%%%%%%%%%%%%%%%%%
%                                   Konfiguration
%
% Hier können alle relevanten Einstellungen für diese Arbeit gesetzt werden.

%%%%%%%%%%%%%%%%%%%%%%%%%%%%%%%%%%%%%%%%%%%%%%%%%%%%%%%%%%%%%%%%%%%%%%%%%%%%%%%

%%%%%%%%%%%%%%%%%%%%%%%%%%%%%%%%%%% Angaben  %%%%%%%%%%%%%%%%%%%%%%%%%%%%%%%%%%%
%% Die meisten der folgenden Daten werden auf dem
%% Deckblatt angezeigt, einige auch im weiteren Verlauf
%% des Dokuments.
\setzekurs{WI16B}
\setzearbeitstitel{Methoden zur Messung des Reifegrads des Prozessmanagements und Ermittlung des Handlungsbedarfs}
\setzestudiengang{Wirtschaftsinformatik - Process Engineering und Digital Management}
\setzedhbw{Villingen-Schwenningen}
\setzebetreuer{Prof. Dr. Puchan}
\setzearbeit{Seminararbeit}
\setzeautor{Steffen Karrer, Markus Kluge, \\ &  & Wladislaw Galster und Florian Löchle}
\sperrvermerkfalse
%%%%%%%%%%%%%%%%%%%%%%%%%%%%%%%%%%%%%%%%%%%%%%%%%%%%%%%%%%%%%%%%%%%%%%%%%%%%%%%%

%%%%%%%%%%%%%%%%%%%%%%%%%%%% Literaturverzeichnis %%%%%%%%%%%%%%%%%%%%%%%%%%%%%%
\newcommand{\insertbibsource}{
	\addbibresource{Bib/Reifegradmodelle.bib}
}

%% Zitierstil
%% siehe: http://ctan.mirrorcatalogs.com/macros/latex/contrib/biblatex/doc/biblatex.pdf (3.3.1 Citation Styles)
%% mögliche Werte z.B numeric-comp, alphabetic, authoryear
\setzezitierstil{numeric-comp}
%%%%%%%%%%%%%%%%%%%%%%%%%%%%%%%%%%%%%%%%%%%%%%%%%%%%%%%%%%%%%%%%%%%%%%%%%%%%%%%%

%%%%%%%%%%%%%%%%%%%%%%%%%%%%%%%%% Layout %%%%%%%%%%%%%%%%%%%%%%%%%%%%%%%%%%%%%%%
%% Verschiedene Schriftarten
% laut nag Warnung: palatino obsolete, use mathpazo, helvet (option scaled=.95), courier instead
\setzeschriftart{helvet} % palatino oder goudysans, lmodern, libertine

%% Paket um Seite im Querformat anzuzeigen
%\usepackage{lscape}

%% Seitenränder
\setzeseitenrand{}

%% Abstand vor Kapitelüberschriften zum oberen Seitenrand
\setzekapitelabstand{22pt}

%% Spaltenabstand
\setzespaltenabstand{10pt}

% Zeilenabstand innerhalb einer Tabelle
\setzezeilenabstand{1.5}

% Absatzabstand
\setzeabsatzabstand{6pt}

%% Überschriften Konfiguration
\newcommand{\configureHeadings}{%
	\titleformat{\section}[hang]
	{\Large\bfseries}
	{\thesection}{1em}{}
	\titleformat{\subsection}[hang]
	{\large\bfseries}
	{\thesubsection}{1em}{}
	\titleformat{\subsubsection}[hang]
	{\normalsize\bfseries}
	{\thesubsubsection}{1em}{}

	\titlespacing
	{\section}
	{0pt}{36pt}{0pt}
	\titlespacing
	{\subsection}
	{0pt}{24pt}{0pt}
	\titlespacing
	{\subsubsection
	}{0pt}{12pt}{0pt}
}
%%%%%%%%%%%%%%%%%%%%%%%%%%%%%%%%%%%%%%%%%%%%%%%%%%%%%%%%%%%%%%%%%%%%%%%%%%%%%%%%

%%%%%%%%%%%%%%%%%%%%%%%%%%%%% Verschiedenes  %%%%%%%%%%%%%%%%%%%%%%%%%%%%%%%%%%%
%% Farben
\newcommand{\ladefarben}{%
	\definecolor{LinkColor}{HTML}{00007A}
	\definecolor{lightgrey}{RGB}{114, 114, 114}
	\definecolor{lightgray}{rgb}{.9,.9,.9}
  \definecolor{darkgray}{rgb}{.4,.4,.4}
	\definecolor{purple}{rgb}{0.65, 0.12, 0.82}
	\definecolor{mygreen}{rgb}{0,0.6,0}
}
%% Mathematikpakete benutzen (Pakete aktivieren)
%\usepackage{amsmath}
%\usepackage{amssymb}
%%%%%%%%%%%%%%%%%%%%%%%%%%%%%%%%%%%%%%%%%%%%%%%%%%%%%%%%%%%%%%%%%%%%%%%%%%%%%%%%

%%%%%%%%%%%%%%%%%%%%%%%%%%%%%%%% Eigenes %%%%%%%%%%%%%%%%%%%%%%%%%%%%%%%%%%%%%%%
%% Hier können Ergänzungen zur Präambel vorgenommen werden (eigene Pakete, Einstellungen)
\newcommand{\spacesmall}{7px}
\newcommand{\spacemedium}{15px}
\newcommand{\spacebig}{30px}



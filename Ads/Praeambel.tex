%**********************
% Dokumenteneinstellung
%**********************
\documentclass[hidelinks,12pt,a4paper,titlepage]{article}

% Einstellungen laden

\newcommand{\setting}[1]{%
	\expandafter\newcommand\csname #1\endcsname{}
	\expandafter\newcommand\csname setze#1\endcsname[1]{\expandafter\renewcommand\csname#1\endcsname{##1}}
}

% Verfügbare Einstellungen definieren

\newif\ifsperrvermerk
\setting{arbeitstitel}
\setting{kurs}
\setting{abschluss}
\setting{studiengang}
\setting{dhbw}
\setting{betreuer}
\setting{arbeit}
\setting{autor}
\setting{sprache}
\setting{schriftart}
\setting{seitenrand}
\setting{kapitelabstand}
\setting{spaltenabstand}
\setting{absatzabstand}
\setting{zeilenabstand}
\setting{zitierstil}
\setting{sperrvermerk}

% Einstellungen lesen

%%%%%%%%%%%%%%%%%%%%%%%%%%%%%%%%%%%%%%%%%%%%%%%%%%%%%%%%%%%%%%%%%%%%%%%%%%%%%%%
%                                   Konfiguration
%
% Hier können alle relevanten Einstellungen für diese Arbeit gesetzt werden.

%%%%%%%%%%%%%%%%%%%%%%%%%%%%%%%%%%%%%%%%%%%%%%%%%%%%%%%%%%%%%%%%%%%%%%%%%%%%%%%

%%%%%%%%%%%%%%%%%%%%%%%%%%%%%%%%%%% Angaben  %%%%%%%%%%%%%%%%%%%%%%%%%%%%%%%%%%%
%% Die meisten der folgenden Daten werden auf dem
%% Deckblatt angezeigt, einige auch im weiteren Verlauf
%% des Dokuments.
\setzekurs{WI16B}
\setzearbeitstitel{Methoden zur Messung des Reifegrads des Prozessmanagements und Ermittlung des Handlungsbedarfs}
\setzestudiengang{Wirtschaftsinformatik - Process Engineering und Digital Management}
\setzedhbw{Villingen-Schwenningen}
\setzebetreuer{Prof. Dr. Puchan}
\setzearbeit{Seminararbeit}
\setzeautor{Steffen Karrer, Markus Kluge, \\ &  & Wladislaw Galster und Florian Löchle}
\sperrvermerkfalse
%%%%%%%%%%%%%%%%%%%%%%%%%%%%%%%%%%%%%%%%%%%%%%%%%%%%%%%%%%%%%%%%%%%%%%%%%%%%%%%%

%%%%%%%%%%%%%%%%%%%%%%%%%%%% Literaturverzeichnis %%%%%%%%%%%%%%%%%%%%%%%%%%%%%%
\newcommand{\insertbibsource}{
	\addbibresource{Bib/Reifegradmodelle.bib}
}

%% Zitierstil
%% siehe: http://ctan.mirrorcatalogs.com/macros/latex/contrib/biblatex/doc/biblatex.pdf (3.3.1 Citation Styles)
%% mögliche Werte z.B numeric-comp, alphabetic, authoryear
\setzezitierstil{numeric-comp}
%%%%%%%%%%%%%%%%%%%%%%%%%%%%%%%%%%%%%%%%%%%%%%%%%%%%%%%%%%%%%%%%%%%%%%%%%%%%%%%%

%%%%%%%%%%%%%%%%%%%%%%%%%%%%%%%%% Layout %%%%%%%%%%%%%%%%%%%%%%%%%%%%%%%%%%%%%%%
%% Verschiedene Schriftarten
% laut nag Warnung: palatino obsolete, use mathpazo, helvet (option scaled=.95), courier instead
\setzeschriftart{helvet} % palatino oder goudysans, lmodern, libertine

%% Paket um Seite im Querformat anzuzeigen
%\usepackage{lscape}

%% Seitenränder
\setzeseitenrand{}

%% Abstand vor Kapitelüberschriften zum oberen Seitenrand
\setzekapitelabstand{22pt}

%% Spaltenabstand
\setzespaltenabstand{10pt}

% Zeilenabstand innerhalb einer Tabelle
\setzezeilenabstand{1.5}

% Absatzabstand
\setzeabsatzabstand{6pt}

%% Überschriften Konfiguration
\newcommand{\configureHeadings}{%
	\titleformat{\section}[hang]
	{\Large\bfseries}
	{\thesection}{1em}{}
	\titleformat{\subsection}[hang]
	{\large\bfseries}
	{\thesubsection}{1em}{}
	\titleformat{\subsubsection}[hang]
	{\normalsize\bfseries}
	{\thesubsubsection}{1em}{}

	\titlespacing
	{\section}
	{0pt}{36pt}{0pt}
	\titlespacing
	{\subsection}
	{0pt}{24pt}{0pt}
	\titlespacing
	{\subsubsection
	}{0pt}{12pt}{0pt}
}
%%%%%%%%%%%%%%%%%%%%%%%%%%%%%%%%%%%%%%%%%%%%%%%%%%%%%%%%%%%%%%%%%%%%%%%%%%%%%%%%

%%%%%%%%%%%%%%%%%%%%%%%%%%%%% Verschiedenes  %%%%%%%%%%%%%%%%%%%%%%%%%%%%%%%%%%%
%% Farben
\newcommand{\ladefarben}{%
	\definecolor{LinkColor}{HTML}{00007A}
	\definecolor{lightgrey}{RGB}{114, 114, 114}
	\definecolor{lightgray}{rgb}{.9,.9,.9}
  \definecolor{darkgray}{rgb}{.4,.4,.4}
	\definecolor{purple}{rgb}{0.65, 0.12, 0.82}
	\definecolor{mygreen}{rgb}{0,0.6,0}
}
%% Mathematikpakete benutzen (Pakete aktivieren)
%\usepackage{amsmath}
%\usepackage{amssymb}
%%%%%%%%%%%%%%%%%%%%%%%%%%%%%%%%%%%%%%%%%%%%%%%%%%%%%%%%%%%%%%%%%%%%%%%%%%%%%%%%

%%%%%%%%%%%%%%%%%%%%%%%%%%%%%%%% Eigenes %%%%%%%%%%%%%%%%%%%%%%%%%%%%%%%%%%%%%%%
%% Hier können Ergänzungen zur Präambel vorgenommen werden (eigene Pakete, Einstellungen)
\newcommand{\spacesmall}{7px}
\newcommand{\spacemedium}{15px}
\newcommand{\spacebig}{30px}




% Sprachen und Sonderzeichen einstellen

% Deutsche Sonderzeichen benutzen
% https://ctan.org/pkg/babel
\usepackage[ngerman]{babel}


%%%%% PAKETE LADEN %%%%%

% Grafiken aus PNG, JPG... Dateien einbinden
% https://ctan.org/pkg/graphicx?lang=de
\usepackage{graphicx}
\graphicspath{{Abbildungen/}}

% Used to wrap a figure in text
\usepackage{wrapfig}

% Latin Modern
\usepackage{helvet}

% Einfügen von Dummy Text zum Testen
%\usepackage{lipsum}

% Umlaute unter UTF8 nutzen
% https://ctan.org/pkg/inputenc
\usepackage[utf8]{inputenc}

% Zeichenencoding
% https://ctan.org/pkg/fontenc
\usepackage[T1]{fontenc}

% Floatende Bilder ermöglichen
% https://ctan.org/pkg/float?lang=de
\usepackage{float}

% Festlegen der Section Header und größen
\usepackage{titlesec}
\configureHeadings

% Packet für Seitenrandabständex und Einstellung für Seitenränder
\usepackage[a4paper,lmargin={4cm},rmargin={1.5cm},tmargin={2.5cm},bmargin={2cm}]{geometry}

% Bricht lange URLs "schoen" um
\usepackage[hyphens,obeyspaces,spaces]{url}

% Official Euro symbol
\usepackage[official]{eurosym}

% Required by biblatex
\usepackage[autostyle=true]{csquotes}

% Paket für Textfarben
\usepackage{xcolor}
\ladefarben

\usepackage[bottom]{footmisc}

\usepackage{array}
\usepackage{longtable}

% Erzeugt Inhaltsverzeichnis mit Querverweisen zu den Kapiteln (PDF Version)
\usepackage{hyperref}

% Neue Kopfzeilen mit FancyHDR
\usepackage{fancyhdr} %Paket laden
\pagestyle{fancy} %eigener Seitenstil
\renewcommand{\sectionmark}[1]{\markright{#1}}
\renewcommand{\subsectionmark}[1]{}
\fancyhf{} %alle Kopf- und Fußzeilenfelder bereinigen
\setlength{\headheight}{15pt}
\renewcommand{\sectionmark}[1]{ \markright{#1}{} }
\renewcommand{\headrulewidth}{0.4pt} %obere Trennlinie

% Paket für Zeilenabstand
\usepackage[onehalfspacing]{setspace}

% Util for the non-disclosure notice lettering on the title page
\usepackage{rotating}

% Paket für die Verwendung von mehreren Spalten
\usepackage{multicol}

% Für die Erstellung des Abkürungszverzeichniss
\usepackage[printonlyused]{acronym}

% Für das Erscheinungsbild von Code-Blocks
%\usepackage{listings}
%\listingsettings
%\renewcommand{\lstlistingname}{Abb.}% Listing -> Algorithm

%%%%% CONFIGUGURATION %%%%%

\usepackage{\schriftart}
\renewcommand{\familydefault}{\sfdefault}

% Titel, Autor und Datum
\title{\titel}
\author{\autor}
\date{\datum}

% Absatzabstände
\setlength{\parindent}{0pt}
\setlength{\parskip}{\absatzabstand}

% Bibliography Settings
\usepackage[backend=biber,style=iso-authoryear,hyperref=true,autocite=footnote,sortlocale=de_DE,maxnames=3,autolang=other,urldate=iso,date=iso,seconds=true]{biblatex}
%\defbibheading{bibliography}[\refname]{\addsec{Literaturverzeichnis}}
%\defbibheading{lit}{\section*{Literaturverzeichnis}}
% Required changes
%\defbibheading{lit}{\section{Literaturverzeichnis}}
\defbibheading{lit}{\section*{Literaturverzeichnis}}
\insertbibsource

% Hurenkinder und Schusterjungen verhindern
% http://projekte.dante.de/DanteFAQ/Silbentrennung
\clubpenalty = 10000 % schließt Schusterjungen aus (Seitenumbruch nach der ersten Zeile eines neuen Absatzes)
\widowpenalty = 10000 % schließt Hurenkinder aus (die letzte Zeile eines Absatzes steht auf einer neuen Seite)
\displaywidowpenalty=10000



% Caption Einstellungen für Bildunterschriften
\usepackage[margin=10pt,textformat=simple,
	labelsep=endash,format=plain,indention=0cm,justification=justified]{caption}
\setlength {\marginparwidth }{2cm}
